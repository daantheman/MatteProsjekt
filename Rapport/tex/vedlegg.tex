\section{Vedlegg}

\subsection{Oppgave 1}
\subsubsection{Oppgave 5.1.21}
Prove the second-order formula for the fourth derivative
\begin{quote}
\begin{equation}\label{eq:oppgave1}
f^{(4)} (x) = \frac{f(x-2h) - 4f(x-h) + 6f(x) - 4f(x+h) + f(x+2h)}{h^4} + O(h^2)
\end{equation}
\end{quote}
Ifølge Taylor's teorem, dersom f er fem ganger kontinuerlig deriverbar, 
kan vi bruke f(x + h) og f(x -h)
\begin{quote}
\begin{equation} \label{eq:f(x+h)}
f(x+h) = f(x) + hf'(x) + \frac{h^2}{2} f''(x) + \frac{h^3}{6} f'''(x) + \frac{h^4}{24} f^4 (x) + \frac{h^5}{120} f^5 (x) + O(h^6)
\end{equation}
\end{quote}
\begin{quote}
\begin{equation} \label{eq:f(x-h)}
f(x-h) = f(x) - hf'(x) + \frac{h^2}{2} f''(x) - \frac{h^3}{6} f'''(x) + \frac{h^4}{24} f^4 (x) - \frac{h^5}{120} f^5 (x) + O(h^6)
\end{equation}
\end{quote}
Legger så sammen f(x+h) + f(x-h) for å eliminere odde-talls deriverte:

\begin{multline} \label{eq:f(x+h)+f(x-h)}
f(x+h) + f(x-h) = \\f(x) + hf'(x) + \frac{h^2}{2} f''(x) + \frac{h^3}{6} f'''(x) + \frac{h^4}{24} f^4 (x) + \frac{h^5}{120} f^5 (x) + O(h^6) +\\
 f(x) - hf'(x) + x\frac{h^2}{2} f''(x) - \frac{h^3}{6} f'''(x) + \frac{h^4}{24} f^4 (x) - \frac{h^5}{120} f^5 (x) + O(h^6) \\ = 2f(x) + h^2 f''(x) + \frac{h^4}{12} f^{(4)} (x) + O (h^6)
\end{multline}
I følge Taylor's teorem, dersom f er fem ganger kontinuerlig deriverbar, kan vi bruke f(x+2h) og f(x-2h). \\
Legger så sammen f(x+2h) + f(x-2h) for å eliminere oddetalls-deriverte. \\
Siden vi allerede har regnet ut f(x+h) + f(x-h), setter vi inn for f(x+2h) og f(x-2h) og får:
\begin{equation} \label{eq:f(x+2h)+f(x-2h)}
f(x+2h) + f(x-2h) = 2f(x) + 2h^2 f(x) + \frac{4h^4}{3} f^4 (x) + O (h^6)
\end{equation}
For å få den fjerde-deriverte aleine, må vi eliminere den andre-deriverte. Dette gjøres ved å gange inn 4 i den første likningen (\ref{eq:f(x+h)+f(x-h)}) og trekker fra likning (\ref{eq:f(x+2h)+f(x-2h)}): 

\begin{multline*}
4 * (f(x+h) + f(x-h)) = 8f(x) + 4h^2 f''(x) + \frac{4h^4}{12} f^4 + 4O(h^6) \\
\Downarrow
\\
4f(x+h) + 4f(x-h) - (f(x+2h) + f(x-2h)) = \\ 
8f(x) + 4h^2 f''(x) + \frac{4h^4}{12} f^4 (x) + 4O(h^6) - \\
2f(x) + 2h^2 f(x) + \frac{4h^4}{3} f^4 (x) + O (h^6) \\
= 6f(x) - h^4 f^4 (x) + 3O (h^6)
\end{multline*}
Snur litt om på likningen og får at:
\begin{multline}
h^4 f^4 = -4f(x+h) - 4f(x-h) + f(x-2h) + f(x-2h) + 6f(x) + 3O(h^6)\\
\Rightarrow f^4 (x) = \frac{f(x-2h) - 4f(x-h) + 6f(x) - 4f(x+h) + f(x+2h)}{h^4} + O(h^2)
\end{multline}


\subsubsection{Oppgave 5.1.22a}
Prove that if f(x) = f'(x) = 0, then
\begin{quote}
\begin{equation}\label{eq:oppgave2}
f^{(4)} (x + h) - \frac{16f(x + h) - 9f(x + 2h) + \frac{8}{3}f(x + 3h) - \frac{1}{4}f(x + 4h)}{h^4} = O(h^2)
\end{equation}
\end{quote}
Starter med å bevise at dersom f(x) = f'(x) = 0, så er
\begin{quote}
\begin{equation}
f(x + h) = 10f(x+h) + 5f(x + 2h) - \frac{5}{3}f(x + 3h) - \frac{1}{4}(x+4h) = O(h^6)
\end{equation}
\end{quote}
Fra oppgave 1, 5.1.21 har vi (\ref{eq:oppgave1}), vi begynner med å skrive denne om til f(x+h) og får:
\begin{quote}
\begin{equation}\label{eq:f^4(x+h)}
f^{(4)}(x+h) = \frac{-4f(x) + f(x-h) + 6f(x+h) - 4f(x+2h) f(x+3h}{h^4} + O(h^2)
\end{equation}
\end{quote}

Nå har vi en likning for $ f^{(4)} (x+h)$ og kan sette denne inn i (\ref{eq:oppgave2})


\begin{multline*}
\frac{-4f(x) + f(x-h) + 6f(x+h) - 4f(x+2h) + f(x+3h)}{h^4} + O(h^2)\\
-
\\
\frac{16f(x+h) - 4f(x+2h) + \frac{8}{3}f(x+3h) - \frac{1}{4}f(x+4h)}{h^4}  = O(h^2)
\\
\Downarrow
\\
\frac{-4f(x)+f(x-h)-10f(x+h)+5f(x+2h)-\frac{5}{3}f(x+3h)+\frac{1}{4}f(x+4h}{h^4} + O(h^2) = O(h^2)
\end{multline*}
Fra oppgaveteksten har vi oppgitt at:
\begin{equation*}
f(x-h)-10f(x+h)+5f(x+2h)-\frac{5}{3}f(x+3h)+\frac{1}{4}f(x+4h) = O(h^6)
\end{equation*}

Vi setter dette inn i likninger vår og får:

\begin{equation*}
\frac{-4f(x) + O(h^6)}{h^4} + O(h^2) = O(h^2)
\end{equation*}

Siden f(x) = f'(x) = 0 ender vi med:


\begin{equation*}
\frac{-0+ O(h^6)}{h^4} + O(h^2) = O(h^2)
\end{equation*}
\begin{equation*}
\Downarrow
\end{equation*}
\begin{equation*}
O(h^2) + O(h^2) = O(h^2)
\end{equation*}

\subsection{Oppgave 2}

lagA.m
\begin{lstlisting}
function A = lagA(n)

e = ones(n,1);
A = spdiags(e*[1 -4 6 -4 1],-2:2,n,n);

A(1,1:4) = [16 -9 8/3 -1/4];
A(n-1,n-3:n) = [16/17 -60/17 72/17 -28/17];
A(n,n-3:n)= [-12/17 96/17 -156/17 72/17];

end
\end{lstlisting}

Scriptet for å kjøre koden:
\begin{lstlisting}
disp('Oppave 2')

A = lagA(10);
full(A)
\end{lstlisting}

\subsection{Oppgave 3}

ebbeam.m
\\
Kan sende med A i fra forrige oppgave, men velger å lage A på nytt slik at dette kan kjøres som et eget skript.
\begin{lstlisting}
% Input:
% E = young's modulus for a meterial
% L = length
% d = diameter
% g = gravity
% w = width
% D = density
function y = ebbeam(E,L,d,D,w,n)

I = w*d^3/12;
h = L/n;
g = 9.81;
b = repmat(-D*w*d*g , n,1) * h^4/(E*I);

A = lagA(n);

y = A\b;
end
\end{lstlisting}
Scriptet for å kjøre koden:
\begin{lstlisting}
disp('Oppgave 3')
format long;
E = 1.3e10;
D = 480;
w = 0.3;
L = 2;
d = 0.03;
n = 10;

disp('Numerisk losning')
y = ebbeam(E,L,d,D,w,n)
\end{lstlisting}
