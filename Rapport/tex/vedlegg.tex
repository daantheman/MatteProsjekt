\section{Vedlegg}

\subsection{Oppgave 1}
\subsubsection{Oppgave 5.1.21}
Prove the second-order formula for the fourth derivative
\begin{quote}
\begin{equation}\label{eq:oppgave1}
f^{(4)} (x) = \frac{f(x-2h) - 4f(x-h) + 6f(x) - 4f(x+h) + f(x+2h)}{h^4} + O(h^2)
\end{equation}
\end{quote}
Ifølge Taylor's teorem, dersom f er fem ganger kontinuerlig deriverbar, 
kan vi bruke f(x + h) og f(x -h)
\begin{quote}
\begin{equation} \label{eq:f(x+h)}
f(x+h) = f(x) + hf'(x) + \frac{h^2}{2} f''(x) + \frac{h^3}{6} f'''(x) + \frac{h^4}{24} f^4 (x) + \frac{h^5}{120} f^5 (x) + O(h^6)
\end{equation}
\end{quote}
\begin{quote}
\begin{equation} \label{eq:f(x-h)}
f(x-h) = f(x) - hf'(x) + \frac{h^2}{2} f''(x) - \frac{h^3}{6} f'''(x) + \frac{h^4}{24} f^4 (x) - \frac{h^5}{120} f^5 (x) + O(h^6)
\end{equation}
\end{quote}
Legger så sammen f(x+h) + f(x-h) for å eliminere odde-talls deriverte:

\begin{multline} \label{eq:f(x+h)+f(x-h)}
f(x+h) + f(x-h) = \\f(x) + hf'(x) + \frac{h^2}{2} f''(x) + \frac{h^3}{6} f'''(x) + \frac{h^4}{24} f^4 (x) + \frac{h^5}{120} f^5 (x) + O(h^6) +\\
 f(x) - hf'(x) + x\frac{h^2}{2} f''(x) - \frac{h^3}{6} f'''(x) + \frac{h^4}{24} f^4 (x) - \frac{h^5}{120} f^5 (x) + O(h^6) \\ = 2f(x) + h^2 f''(x) + \frac{h^4}{12} f^{(4)} (x) + O (h^6)
\end{multline}
I følge Taylor's teorem, dersom f er fem ganger kontinuerlig deriverbar, kan vi bruke f(x+2h) og f(x-2h). \\
Legger så sammen f(x+2h) + f(x-2h) for å eliminere oddetalls-deriverte. \\
Siden vi allerede har regnet ut f(x+h) + f(x-h), setter vi inn for f(x+2h) og f(x-2h) og får:
\begin{equation} \label{eq:f(x+2h)+f(x-2h)}
f(x+2h) + f(x-2h) = 2f(x) + 2h^2 f(x) + \frac{4h^4}{3} f^4 (x) + O (h^6)
\end{equation}
For å få den fjerde-deriverte aleine, må vi eliminere den andre-deriverte. Dette gjøres ved å gange inn 4 i den første likningen (\ref{eq:f(x+h)+f(x-h)}) og trekker fra likning (\ref{eq:f(x+2h)+f(x-2h)}): 

\begin{multline*}
4 * (f(x+h) + f(x-h)) = 8f(x) + 4h^2 f''(x) + \frac{4h^4}{12} f^4 + 4O(h^6) \\
\Downarrow
\\
4f(x+h) + 4f(x-h) - (f(x+2h) + f(x-2h)) = \\ 
8f(x) + 4h^2 f''(x) + \frac{4h^4}{12} f^4 (x) + 4O(h^6) - \\
2f(x) + 2h^2 f(x) + \frac{4h^4}{3} f^4 (x) + O (h^6) \\
= 6f(x) - h^4 f^4 (x) + 3O (h^6)
\end{multline*}
Snur litt om på likningen og får at:
\begin{multline}
h^4 f^4 = -4f(x+h) - 4f(x-h) + f(x-2h) + f(x-2h) + 6f(x) + 3O(h^6)\\
\Rightarrow f^4 (x) = \frac{f(x-2h) - 4f(x-h) + 6f(x) - 4f(x+h) + f(x+2h)}{h^4} + O(h^2)
\end{multline}


\subsubsection{Oppgave 5.1.22a}
Prove that if f(x) = f'(x) = 0, then
\begin{quote}
\begin{equation}\label{eq:oppgave2}
f^{(4)} (x + h) - \frac{16f(x + h) - 9f(x + 2h) + \frac{8}{3}f(x + 3h) - \frac{1}{4}f(x + 4h)}{h^4} = O(h^2)
\end{equation}
\end{quote}
Starter med å bevise at dersom f(x) = f'(x) = 0, så er
\begin{quote}
\begin{equation}
f(x + h) = 10f(x+h) + 5f(x + 2h) - \frac{5}{3}f(x + 3h) - \frac{1}{4}(x+4h) = O(h^6)
\end{equation}
\end{quote}
Fra oppgave 1, 5.1.21 har vi (\ref{eq:oppgave1}), vi begynner med å skrive denne om til f(x+h) og får:
\begin{quote}
\begin{equation}\label{eq:f^4(x+h)}
f^{(4)}(x+h) = \frac{-4f(x) + f(x-h) + 6f(x+h) - 4f(x+2h) f(x+3h}{h^4} + O(h^2)
\end{equation}
\end{quote}

Nå har vi en likning for $ f^{(4)} (x+h)$ og kan sette denne inn i (\ref{eq:oppgave2})


\begin{multline*}
\frac{-4f(x) + f(x-h) + 6f(x+h) - 4f(x+2h) + f(x+3h)}{h^4} + O(h^2)\\
-
\\
\frac{16f(x+h) - 4f(x+2h) + \frac{8}{3}f(x+3h) - \frac{1}{4}f(x+4h)}{h^4}  = O(h^2)
\\
\Downarrow
\\
\frac{-4f(x)+f(x-h)-10f(x+h)+5f(x+2h)-\frac{5}{3}f(x+3h)+\frac{1}{4}f(x+4h}{h^4} + O(h^2) = O(h^2)
\end{multline*}
Fra oppgaveteksten har vi oppgitt at:
\begin{equation*}
f(x-h)-10f(x+h)+5f(x+2h)-\frac{5}{3}f(x+3h)+\frac{1}{4}f(x+4h) = O(h^6)
\end{equation*}

Vi setter dette inn i likninger vår og får:

\begin{equation*}
\frac{-4f(x) + O(h^6)}{h^4} + O(h^2) = O(h^2)
\end{equation*}

Siden f(x) = f'(x) = 0 ender vi med:


\begin{equation*}
\frac{-0+ O(h^6)}{h^4} + O(h^2) = O(h^2)
\end{equation*}
\begin{equation*}
\Downarrow
\end{equation*}
\begin{equation*}
O(h^2) + O(h^2) = O(h^2)
\end{equation*}

\subsection{Oppgave 6}
\subsubsection{Oppgave 6a}
Vi legger til en funksjon i Euler-likningen
\begin{quote}
$s(x) = -pgsin\frac{\pi}{L} x $
\end{quote} 
til kraftdelen til f(x)
\begin{quote}
$EIy^{(4)} = f(x) + s(x)$
\end{quote}
Skal bevise at
\begin{quote}
\begin{equation}
y(x) = \frac{f}{24EI} x^2 (x^2 - 4Lx + 6L^2) - \frac{gpL}{EI\pi} (\frac{L^3}{\pi^3} sin\frac{\pi}{L}x - \frac{x^3}{6} + \frac{Lx^2}{2} - \frac{L^2 x}{\pi^2})
\end{equation}
\end{quote}
tilfredstiller Euler-Bernoullie likningen og randbetingelsene for en bjelke som er fast i den ene enden og fri i den andre\\
$y(0) = y'(0) = y''(L) = y'''(L) = 0$ \\
\\
1. Starter med å bevise at y(0)= 0
\begin{quote}
\begin{multline*}
y(0) = \frac{f}{24EI} 0^2 (0^2 - 4L0 + 6L^2) - \frac{gpL}{EI\pi} (\frac{L^3}{\pi^3} sin\frac{\pi}{L}x - \frac{0^3}{6} + \frac{L0^2}{2} - \frac{L^2 0}{\pi^2}) \\
y(0) = 0 - \frac{gpL}{EI\pi} (\frac{L^3}{\pi^3} sin (0) - 0 + 0 - 0 ) \\
y(0) = 0
\end{multline*}
\end{quote}
Første kriterie er oppfylt.
\\
2. Skal nå bevise kriterie 2, at y'(0) = 0, starter med å finne den deriverte.
\begin{quote}
\begin{multline}
y'(x) = \frac{d}{dx} [\frac{fx^2(x^2-4Lx+6L^2)}{24EI} - \frac{gpL}{EI\pi} (\frac{L^3}{\pi^3} sin \frac{\pi}{L}x - \frac{x^3}{6} + \frac{Lx^2}{2} - \frac{L^2x}{\pi^2})]
\end{multline}
\end{quote}

\begin{quote}
\begin{multline*}
y'(x) = \frac{fx(x^2-3Lx+3L^2)}{6EI} - \frac{gpL}{EI\pi}*\frac{d}{dx} (\frac{L^3}{\pi^3} sin \frac{\pi}{L}x - \frac{x^3}{6} + \frac{Lx^2}{2} - \frac{L^2x}{\pi^2})
\end{multline*}
\end{quote}

\begin{quote}
\begin{multline*}
y'(x) = \frac{fx(x^2-3Lx+3L^2)}{6EI} - \frac{gpL}{EI\pi} (\frac{L^2}{\pi^2} cos(\frac{\pi}{L}x) - \frac{x^2}{2} + Lx - \frac{L^2}{\pi^2})
\end{multline*}
\end{quote}
Sjekker om y'(0) = 0
\begin{quote}
\begin{multline*}
y'(0) = \frac{f0(0^2-3L0+3L^2)}{6EI} - \frac{gpL}{EI\pi} (\frac{L^2}{\pi^2} cos(\frac{\pi}{L}0) - \frac{0^2}{2} + L0 - \frac{L^2}{\pi^2})
\end{multline*}
\end{quote}

\begin{quote}
\begin{equation*}
y'(0) = 0 - \frac{gpL}{EI\pi} (\frac{L^2}{\pi^2} * 1 -\frac{L^2}{\pi^2}) \\
\end{equation*}
\end{quote}

\begin{quote}
\begin{equation*}
y'(0) = 0 - \frac{gpL}{EI\pi} (0) = 0
\end{equation*}
\end{quote}
Andre kriteriet er oppfylt.
\\
3. Skal nå bevise kriteriet 3, y''(L) = 0. Starter med å finne den andre deriverte.
\begin{quote}
\begin{multline*}
y'(x) = \frac{fx(x^2-3Lx+3L^2)}{6EI} - \frac{gpL}{EI\pi} (\frac{L^2}{\pi^2} cos(\frac{\pi}{L}x) - \frac{x^2}{2} + Lx - \frac{L^2}{\pi^2})
\end{multline*}
\end{quote}
Fra oppgave 4a, vet vi at
\begin{quote}
\begin{multline*}
y''(x) = \frac{f(x-L)^2}{2EI} - \frac{gpL}{EI\pi} \frac{d}{dx}(\frac{L^2}{\pi^2} cos(\frac{\pi}{L}x) - \frac{x^2}{2} + Lx - \frac{L^2}{\pi^2})
\end{multline*}
\end{quote}

\begin{quote}
\begin{multline}
y''(x) = \frac{f(x-L)^2}{2EI} - \frac{gpL}{EI\pi}* [\frac{L}{\pi} (-sin(\frac{\pi}{L}x)) - x + L]
\end{multline}
\end{quote}
Sjekker nå om y''(L)=0
\begin{quote}
\begin{multline}
y''(L) = \frac{f(L-L)^2}{2EI} - \frac{gpL}{EI\pi}* [\frac{L}{\pi} (-sin(\frac{\pi}{L}L)) - L + L]
\end{multline}
\end{quote}

\begin{quote}
\begin{equation*}
y''(L) = 0 - \frac{gpL}{EI\pi}* [\frac{L}{\pi} (-sin(\pi)) ]
\end{equation*}
\end{quote}

\begin{quote}
\begin{equation*}
y''(L) = 0 - \frac{gpL}{EI\pi}* [\frac{L}{\pi} (0) ]
\end{equation*}
\end{quote}

\begin{quote}
\begin{equation*}
y''(L) = 0 - 0 = 0
\end{equation*}
\end{quote}
Tredje kriteriet stemmer da y''(L) = 0.
4. Skal nå bevise at fjerde kriteriet, y'''(L) = 0, og starter med å finne den tredje deriverte
\begin{quote}
\begin{multline*}
y''(x) = \frac{f(x-L)^2}{2EI} - \frac{gpL}{EI\pi}(\frac{L^2}{\pi^2} cos(\frac{\pi}{L}x) - \frac{x^2}{2} + Lx - \frac{L^2}{\pi^2})
\end{multline*}
\end{quote}

\begin{quote}
\begin{multline*}
y'''(x) = \frac{f(x-L)^2}{2EI} - \frac{gpL}{EI\pi} \frac{d}{dx}(\frac{L^2}{\pi^2} cos(\frac{\pi}{L}x) - \frac{x^2}{2} + Lx - \frac{L^2}{\pi^2})
\end{multline*}
\end{quote}

\begin{quote}
\begin{equation}
y'''(x) = \frac{f(x-L)}{EI} - \frac{gpL}{EI\pi} (-cos(\frac{\pi}{L}x) - 1)
\end{equation}
\end{quote}
Sjekker så tredje kriteriet, y'''(L) = 0
\begin{quote}
\begin{equation*}
y'''(L) = \frac{f(L-L)}{EI} - \frac{gpL}{EI\pi} (-cos(\frac{\pi}{L}L) - 1)
\end{equation*}
\end{quote}

\begin{quote}
\begin{equation*}
y'''(L) = 0 - \frac{gpL}{EI\pi} (1 - 1)
\end{equation*}
\end{quote}

\begin{quote}
\begin{equation*}
y'''(L) = 0 - \frac{gpL}{EI\pi} 0 = 0
\end{equation*}
\end{quote}
Fjerde kriteriet stemmer fordi y'''(L) = 0.
