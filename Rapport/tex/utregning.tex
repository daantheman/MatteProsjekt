\section{Utregning}

\subsection{Oppgave 1}
\subsubsection{Oppgave 5.1.21}
Prove the second-order formula for the fourth derivative
\begin{quote}
\begin{equation}\label{eq:oppgave1}
f^{(4)} (x) = \frac{f(x-2h) - 4f(x-h) + 6f(x) - 4f(x+h) + f(x+2h)}{h^4} + O(h^2)
\end{equation}
\end{quote}
Ifølge Taylor's teorem, dersom f er fem ganger kontinuerlig deriverbar, 
kan vi bruke f(x + h) og f(x -h)
\begin{quote}
\begin{equation} \label{eq:f(x+h)}
f(x+h) = f(x) + hf'(x) + \frac{h^2}{2} f''(x) + \frac{h^3}{6} f'''(x) + \frac{h^4}{24} f^4 (x) + \frac{h^5}{120} f^5 (x) + O(h^6)
\end{equation}
\end{quote}
\begin{quote}
\begin{equation} \label{eq:f(x-h)}
f(x-h) = f(x) - hf'(x) + \frac{h^2}{2} f''(x) - \frac{h^3}{6} f'''(x) + \frac{h^4}{24} f^4 (x) - \frac{h^5}{120} f^5 (x) + O(h^6)
\end{equation}
\end{quote}
Legger så sammen f(x+h) + f(x-h) for å eliminere odde-talls deriverte:

\begin{multline} \label{eq:f(x+h)+f(x-h)}
f(x+h) + f(x-h) = \\f(x) + hf'(x) + \frac{h^2}{2} f''(x) + \frac{h^3}{6} f'''(x) + \frac{h^4}{24} f^4 (x) + \frac{h^5}{120} f^5 (x) + O(h^6) +\\
 f(x) - hf'(x) + x\frac{h^2}{2} f''(x) - \frac{h^3}{6} f'''(x) + \frac{h^4}{24} f^4 (x) - \frac{h^5}{120} f^5 (x) + O(h^6) \\ = 2f(x) + h^2 f''(x) + \frac{h^4}{12} f^{(4)} (x) + O (h^6)
\end{multline}
I følge Taylor's teorem, dersom f er fem ganger kontinuerlig deriverbar, kan vi bruke f(x+2h) og f(x-2h). \\
Legger så sammen f(x+2h) + f(x-2h) for å eliminere oddetalls-deriverte. \\
Siden vi allerede har regnet ut f(x+h) + f(x-h), setter vi inn for f(x+2h) og f(x-2h) og får:
\begin{equation} \label{eq:f(x+2h)+f(x-2h)}
f(x+2h) + f(x-2h) = 2f(x) + 2h^2 f(x) + \frac{4h^4}{3} f^4 (x) + O (h^6)
\end{equation}
For å få den fjerde-deriverte aleine, må vi eliminere den andre-deriverte. Dette gjøres ved å gange inn 4 i den første likningen (\ref{eq:f(x+h)+f(x-h)}) og trekker fra likning (\ref{eq:f(x+2h)+f(x-2h)}): 

\begin{multline*}
4 * (f(x+h) + f(x-h)) = 8f(x) + 4h^2 f''(x) + \frac{4h^4}{12} f^4 + 4O(h^6) \\
\Downarrow
\\
4f(x+h) + 4f(x-h) - (f(x+2h) + f(x-2h)) = \\ 
8f(x) + 4h^2 f''(x) + \frac{4h^4}{12} f^4 (x) + 4O(h^6) - \\
2f(x) + 2h^2 f(x) + \frac{4h^4}{3} f^4 (x) + O (h^6) \\
= 6f(x) - h^4 f^4 (x) + 3O (h^6)
\end{multline*}
Snur litt om på likningen og får at:
\begin{multline}
h^4 f^4 = -4f(x+h) - 4f(x-h) + f(x-2h) + f(x-2h) + 6f(x) + 3O(h^6)\\
\Rightarrow f^4 (x) = \frac{f(x-2h) - 4f(x-h) + 6f(x) - 4f(x+h) + f(x+2h)}{h^4} + O(h^2)
\end{multline}


\subsubsection{Oppgave 5.1.22a}
Prove that if f(x) = f'(x) = 0, then
\begin{quote}
\begin{equation}\label{eq:oppgave2}
f^{(4)} (x + h) - \frac{16f(x + h) - 9f(x + 2h) + \frac{8}{3}f(x + 3h) - \frac{1}{4}f(x + 4h)}{h^4} = O(h^2)
\end{equation}
\end{quote}
Starter med å bevise at dersom f(x) = f'(x) = 0, så er
\begin{quote}
\begin{equation}
f(x + h) - 10f(x+h) + 5f(x + 2h) - \frac{5}{3}f(x + 3h) - \frac{1}{4}(x+4h) = O(h^6)
\end{equation}
\end{quote}
Fra oppgave 1, 5.1.21 har vi (\ref{eq:oppgave1}), vi begynner med å skrive denne om til f(x+h) og får:
\begin{quote}
\begin{equation}\label{eq:f^4(x+h)}
f^{(4)}(x+h) = \frac{-4f(x) + f(x-h) + 6f(x+h) - 4f(x+2h) f(x+3h}{h^4} + O(h^2)
\end{equation}
\end{quote}

Nå har vi en likning for $ f^{(4)} (x+h)$ og kan sette denne inn i (\ref{eq:oppgave2})


\begin{multline*}
\frac{-4f(x) + f(x-h) + 6f(x+h) - 4f(x+2h) + f(x+3h)}{h^4} + O(h^2)\\
-
\\
\frac{16f(x+h) - 4f(x+2h) + \frac{8}{3}f(x+3h) - \frac{1}{4}f(x+4h)}{h^4}  = O(h^2)
\\
\Downarrow
\\
\frac{-4f(x)+f(x-h)-10f(x+h)+5f(x+2h)-\frac{5}{3}f(x+3h)+\frac{1}{4}f(x+4h}{h^4} + O(h^2) = O(h^2)
\end{multline*}
Fra oppgaveteksten har vi oppgitt at:
\begin{equation*}
f(x-h)-10f(x+h)+5f(x+2h)-\frac{5}{3}f(x+3h)+\frac{1}{4}f(x+4h) = O(h^6)
\end{equation*}

Vi setter dette inn i likninger vår og får:

\begin{equation*}
\frac{-4f(x) + O(h^6)}{h^4} + O(h^2) = O(h^2)
\end{equation*}

Siden f(x) = f'(x) = 0 ender vi med:


\begin{equation*}
\frac{-0+ O(h^6)}{h^4} + O(h^2) = O(h^2)
\end{equation*}
\begin{equation*}
\Downarrow
\end{equation*}
\begin{equation*}
O(h^2) + O(h^2) = O(h^2)
\end{equation*}

\subsection{Oppgave 2}

lagA.m
\begin{lstlisting}
function A = lagA(n)

e = ones(n,1);
A = spdiags(e*[1 -4 6 -4 1],-2:2,n,n);

A(1,1:4) = [16 -9 8/3 -1/4];
A(n-1,n-3:n) = [16/17 -60/17 72/17 -28/17];
A(n,n-3:n)= [-12/17 96/17 -156/17 72/17];

end
\end{lstlisting}

Scriptet for å kjøre koden:
\begin{lstlisting}
disp('Oppave 2')

A = lagA(10);
full(A)
\end{lstlisting}

\subsection{Oppgave 3}

ebbeam.m
\\
Kan sende med A i fra forrige oppgave, men velger å lage A på nytt slik at dette kan kjøres som et eget skript.
\begin{lstlisting}
% Input:
% E = young's modulus for a meterial
% L = length
% d = diameter
% g = gravity
% w = width
% D = density
function y = ebbeam(E,L,d,D,w,n)

I = w*d^3/12;
h = L/n;
g = 9.81;
b = repmat(-D*w*d*g , n,1) * h^4/(E*I);

A = lagA(n);

y = A\b;
end
\end{lstlisting}
Scriptet for å kjøre koden:
\begin{lstlisting}
disp('Oppgave 3')
format long;
E = 1.3e10;
D = 480;
w = 0.3;
L = 2;
d = 0.03;
n = 10;

disp('Numerisk losning')
y = ebbeam(E,L,d,D,w,n)
\end{lstlisting}

\subsection{Oppgave 4}
\subsubsection{Oppgave 4a}
Vi har at Euler-Bernouillilikningen,
\begin{quote}
\begin{equation}
EIy'''' = f(x)
\end{equation}
\end{quote}
er oppfylt av y(x), som er den vertikale forskyvningen av en L meter lang bjelke. Den korrekte løsningen av likningen med konstanten f(x) = f er;
\begin{quote}
\begin{equation}
y(x)=(f/24EI) x^2 (x^2-4Lx+6L^2)
\end{equation}
\end{quote}
Skal nå vise at y(x) er den korrekte løsningen ved å derivere den fire ganger og sette inn i Euler-Bernouillilikningen. Begynner med å finne den deriverte med hensyn på x. Vi behandler f/EI og L som konstanter.

Begynner med å finne den deriverte:
\begin{quote}
\begin{equation*}
y=(f/24EI)(x^4-4Lx^3+6L^2x^2)=
\end{equation*}
\begin{equation*}
y'=\frac{d}{dx}[(f/24EI)]\frac{d}{dx}[x^4-4Lx^3+6L^2x^2]=
\end{equation*}
\begin{equation*}
y'=(f/24EI)(4x^3-12Lx^2+12L^2x)=
\end{equation*}
\end{quote}
Her er den derivert en gang, fortsetter med å finne den andrederiverte:
\begin{quote}
\begin{equation*}
y''=\frac{d}{dx}[(f/24EI)]\frac{d}{dx}[4x^3-12Lx^2+12L^2x]=
\end{equation*}
\begin{equation*}
y''=(f/24EI)(12x^2-24Lx+12L^2)=
\end{equation*}
\end{quote}
Her har vi funnet den andrederiverte, fortsetter med å finne den tredjederiverte;
\begin{quote}
\begin{equation*}
y'''=\frac{d}{dx}[(f/24EI)]\frac{d}{dx}[12x^2-24Lx+12L^2]=
\end{equation*}
\begin{equation*}
y'''=(f/24EI)(24x-24L+0)=
\end{equation*}
\end{quote}
Her har vi funnet den tredjederiverte, fortsetter med å finne den fjerdederiverte;
\begin{quote}
\begin{equation*}
y''''=\frac{d}{dx}[(f/24EI)]\frac{d}{dx}[24x-24L]=
\end{equation*}
\begin{equation*}
y''''=(f/24EI)(24)=
\end{equation*}
\begin{equation*}
y''''=\frac{24f}{24EI}
\end{equation*}
\begin{equation*}
y''''=\frac{f}{EI}
\end{equation*}
\end{quote}
Vi har nå funnet den fjerdederiverte til y(x). Vi kan nå sette inn i Euler-Bernouillilikningen;
\begin{quote}
\begin{equation*}
(1) y'''' = \frac{f(x)}{EI}
\end{equation*}
\begin{equation*}
(2) EIy'''' = f(x)
\end{equation*}
\end{quote}
Setter likning (1) inn i likning (2)
\begin{quote}
\begin{equation*}
EI*\frac{f(x)}{EI} = f(x)
\end{equation*}
\begin{equation*}
f(x) = f(x)
\end{equation*}
\end{quote}
Her har vi vist at y(x) oppfyller likningen ved å derivere fire ganger.

\subsubsection{Oppgave 4b}
Skal vise at for den korrekte løsningen er $y^{(6)}$(c) = 0. Vi vet fra oppgave 4a at den 4. deriverte av den korrekte løsningen er:
\begin{equation*}
y^{(4)}(x) = \frac{f}{EI}
\end{equation*}
Vi kan derfra visa at den 6. deriverte er 0.
\begin{equation}
f^{(5)}(x) = \frac{d}{dx}[\frac{f}{EI}] = 0
\end{equation}
Deriverer m.h.p x og $\frac{f}{EI}$ er en konstant.
\begin{equation}
y^{(6)}(x) = \frac{d}{dx}[0] = 0
\end{equation}
Slik har vi vist at $y^{(6)}(c) = 0$.

\subsubsection{Oppgave 4c}
Bruker her MATLAB til å regne ut den korrekte løsningen $y_e = [y(0.2)\;\;y(0.4)\;\;y(0.6)\;\;...\;\;y(2.0)]^T$ Regner så ut den numerisk fjerdederiverte av $y(x)$ ved hjelp av $Diff_4(y)= Ay$. Her er MATLAB-koden:
\begin{lstlisting}
disp('Oppgave 4c')
[E, I, D, d, w, f, g, L] = hentKonstanter();

% Regner ut den eksakte losningen ye.
ye = zeros(1,10);
count = 1;
for i = (0.2:0.2:2)
    ye(count) = correct_y(f,E,I,L,i);
    count = count + 1;
end
ye = transpose(ye)


% Lager samme A som oppgave 2 og regner ut A*ye.
disp('Regner ut C = Ay (A-matrisen er laget med lagA(10) fra oppgave 2)')
A = lagA(10);
C = A*ye
\end{lstlisting}
Resultatet av denne koden blir som følger:
\begin{lstlisting}
Oppgave 4c

ye =

    -1.806247384615386e-04
    -6.748475076923080e-04
    -1.416986584615385e-03
    -2.349087507692309e-03
    -3.420923076923078e-03
    -4.589993353846155e-03
    -5.821525661538464e-03
    -7.088474584615388e-03
    -8.371521969230772e-03
    -9.659076923076925e-03

Regner ut C = Ay (A-matrisen er laget med lagA(10) fra oppgave 2)

C =

    -7.727261538461477e-06
    -7.727261538462236e-06
    -7.727261538458333e-06
    -7.727261538467874e-06
    -7.727261538454863e-06
    -7.727261538463537e-06
    -7.727261538460067e-06
    -7.727261538465272e-06
    -7.727261538463537e-06
    -7.727261538463537e-06
\end{lstlisting}

\subsubsection{Oppgave 4d}
Her skal vi sammenligne svaret fra oppgave 4c med vektoren $b$ fra oppgave 3 og finne foroverfeil og relativ foroverfeil til $Diff_4(y)= Ay$. Etterpå skal vi anta at relativ bakoverfeil er $\epsilon_{mach} = 2^{-52}$, finne feilforstørrelsen og sammenligne den med kondisjonstallet til $A$. Her er MATLAB-koden:
\begin{lstlisting}
disp('Oppgave 4d')
disp('Sammenligner svaret C fra c) med vektoren b fra oppgave 3')

% Gjenskaper vektoren b fra oppgave 3.
h = L/10;
b = repmat(f, 10, 1) * h^4/(E*I);
table(C, b)

% Finner foroverfeil FE ved aa ta ||C - b||
disp('foroverfeil:')
FE = max( abs(C-b) )

% Finner relativ foroverfeil rFE ved aa ta ||C - b|| / ||C||
disp('Relativ foroverfeil:')
rFE = (FE)/( max( abs(C) ) )

% Antar at relativ bakoverfeil er rBE=2^-52 og regner ut
% feilforstorrelsen som rFE/rBE, sammenligner med kondisjonstallet til A.
disp('Feilforstorrelse:')
rBE = 2^-52;
rFE/rBE
disp('Kondisjonstall til A:')
cond(full(A))
\end{lstlisting}
Resultatet av koden blir som følger:
\begin{lstlisting}
Oppgave 4d
Sammenligner svaret C fra c) med vektoren b fra oppgave 3

ans = 

              C                        b          
    _____________________    _____________________

    -7.72726153846148e-06    -7.72726153846154e-06
    -7.72726153846224e-06    -7.72726153846154e-06
    -7.72726153845833e-06    -7.72726153846154e-06
    -7.72726153846787e-06    -7.72726153846154e-06
    -7.72726153845486e-06    -7.72726153846154e-06
    -7.72726153846354e-06    -7.72726153846154e-06
    -7.72726153846007e-06    -7.72726153846154e-06
    -7.72726153846527e-06    -7.72726153846154e-06
    -7.72726153846354e-06    -7.72726153846154e-06
    -7.72726153846354e-06    -7.72726153846154e-06

foroverfeil:

FE =

     6.678007756152904e-18

Relativ foroverfeil:

rFE =

     8.642140197932253e-13

Feilforstorrelse:

ans =

     3.892073937509128e+03
     

Kondisjonstall til A:

ans =

     1.857401842242595e+04
\end{lstlisting}
Vi vet fra teorien at kondisjonstallet til A er den maksimale feilforstørrelsen man kan oppnå. Med et kondisjonstall på $10^4$ kan man forvente $16-4=12$ korrekte desimaler i svaret. ******* SKRIV MER **********

\subsubsection{Oppgave 4e}
Her skal vi sammenligne svaret fra oppgave 3, $y_c$, med den korrekte løsningen, $y_e$, fra oppgave 4c. Vi skal finne foroverfeilen $||y_c-y_e||_1$. Dette er en 1-norm som regnes ut som summen av absolutt-verdien til alle elementene i vektoren. Her er MATLAB-koden:
\begin{lstlisting}
disp('Oppgave 4e')
disp('Sammenligner den eksakte losningen (ye) med vektoren (yc) fra opgave 3')

yc = ebbeam(L,10,f,E,I);
table(yc, ye)

disp('foroverfeil: ||yc-ye||_1')
% Denne foroverfeilen er en 1-norm, kan regnes paa to maater:
sum(abs(yc-ye));
norm(yc-ye, 1)
\end{lstlisting}
Her er resultatet av koden:
\begin{lstlisting}
Oppgave 4e
Sammenligner den eksakte losningen (ye) med vektoren (yc) fra oppgave 3

ans = 

             yc                       ye          
    _____________________    _____________________

    -1.80624738461544e-04    -1.80624738461539e-04
    -6.74847507692328e-04    -6.74847507692308e-04
    -1.41698658461543e-03    -1.41698658461539e-03
    -2.34908750769237e-03    -2.34908750769231e-03
    -3.42092307692317e-03    -3.42092307692308e-03
    -4.58999335384627e-03    -4.58999335384616e-03
    -5.82152566153860e-03    -5.82152566153846e-03
    -7.08847458461555e-03    -7.08847458461539e-03
    -8.37152196923095e-03    -8.37152196923077e-03
    -9.65907692307711e-03    -9.65907692307692e-03

foroverfeil: ||yc-ye||_1

ans =

     9.980623098815311e-16
\end{lstlisting}
Vi ser at denne foroverfeilen er i samme størrelsesorden som $\epsilon_{mach}$, ******* HVORFOR? **********

\subsection{Oppgave 6}
\subsubsection{Oppgave 6a}
Vi legger til en funksjon i Euler-likningen
\begin{quote}
$s(x) = -pgsin\frac{\pi}{L} x $
\end{quote} 
til kraftdelen til f(x)
\begin{quote}
$EIy^{(4)} = f(x) + s(x)$
\end{quote}
Skal bevise at
\begin{quote}
\begin{equation}
y(x) = \frac{f}{24EI} x^2 (x^2 - 4Lx + 6L^2) - \frac{gpL}{EI\pi} (\frac{L^3}{\pi^3} sin\frac{\pi}{L}x - \frac{x^3}{6} + \frac{Lx^2}{2} - \frac{L^2 x}{\pi^2})
\end{equation}
\end{quote}
tilfredstiller Euler-Bernoullie likningen og randbetingelsene for en bjelke som er fast i den ene enden og fri i den andre\\
$y(0) = y'(0) = y''(L) = y'''(L) = 0$ \\
\\
1. Starter med å bevise at y(0)= 0
\begin{quote}
\begin{multline*}
y(0) = \frac{f}{24EI} 0^2 (0^2 - 4L0 + 6L^2) - \frac{gpL}{EI\pi} (\frac{L^3}{\pi^3} sin\frac{\pi}{L}x - \frac{0^3}{6} + \frac{L0^2}{2} - \frac{L^2 0}{\pi^2}) \\
y(0) = 0 - \frac{gpL}{EI\pi} (\frac{L^3}{\pi^3} sin (0) - 0 + 0 - 0 ) \\
y(0) = 0
\end{multline*}
\end{quote}
Første kriterie er oppfylt.
\\
2. Skal nå bevise kriterie 2, at y'(0) = 0, starter med å finne den deriverte.
\begin{quote}
\begin{multline}
y'(x) = \frac{d}{dx} [\frac{fx^2(x^2-4Lx+6L^2)}{24EI} - \frac{gpL}{EI\pi} (\frac{L^3}{\pi^3} sin \frac{\pi}{L}x - \frac{x^3}{6} + \frac{Lx^2}{2} - \frac{L^2x}{\pi^2})]
\end{multline}
\end{quote}

\begin{quote}
\begin{multline*}
y'(x) = \frac{fx(x^2-3Lx+3L^2)}{6EI} - \frac{gpL}{EI\pi}*\frac{d}{dx} (\frac{L^3}{\pi^3} sin \frac{\pi}{L}x - \frac{x^3}{6} + \frac{Lx^2}{2} - \frac{L^2x}{\pi^2})
\end{multline*}
\end{quote}

\begin{quote}
\begin{multline*}
y'(x) = \frac{fx(x^2-3Lx+3L^2)}{6EI} - \frac{gpL}{EI\pi} (\frac{L^2}{\pi^2} cos(\frac{\pi}{L}x) - \frac{x^2}{2} + Lx - \frac{L^2}{\pi^2})
\end{multline*}
\end{quote}
Sjekker om y'(0) = 0
\begin{quote}
\begin{multline*}
y'(0) = \frac{f0(0^2-3L0+3L^2)}{6EI} - \frac{gpL}{EI\pi} (\frac{L^2}{\pi^2} cos(\frac{\pi}{L}0) - \frac{0^2}{2} + L0 - \frac{L^2}{\pi^2})
\end{multline*}
\end{quote}

\begin{quote}
\begin{equation*}
y'(0) = 0 - \frac{gpL}{EI\pi} (\frac{L^2}{\pi^2} * 1 -\frac{L^2}{\pi^2}) \\
\end{equation*}
\end{quote}

\begin{quote}
\begin{equation*}
y'(0) = 0 - \frac{gpL}{EI\pi} (0) = 0
\end{equation*}
\end{quote}
Andre kriteriet er oppfylt.
\\
3. Skal nå bevise kriteriet 3, y''(L) = 0. Starter med å finne den andre deriverte.
\begin{quote}
\begin{multline*}
y'(x) = \frac{fx(x^2-3Lx+3L^2)}{6EI} - \frac{gpL}{EI\pi} (\frac{L^2}{\pi^2} cos(\frac{\pi}{L}x) - \frac{x^2}{2} + Lx - \frac{L^2}{\pi^2})
\end{multline*}
\end{quote}
Fra oppgave 4a, vet vi at
\begin{quote}
\begin{multline*}
y''(x) = \frac{f(x-L)^2}{2EI} - \frac{gpL}{EI\pi} \frac{d}{dx}(\frac{L^2}{\pi^2} cos(\frac{\pi}{L}x) - \frac{x^2}{2} + Lx - \frac{L^2}{\pi^2})
\end{multline*}
\end{quote}

\begin{quote}
\begin{multline}
y''(x) = \frac{f(x-L)^2}{2EI} - \frac{gpL}{EI\pi}* [\frac{L}{\pi} (-sin(\frac{\pi}{L}x)) - x + L]
\end{multline}
\end{quote}
Sjekker nå om y''(L)=0
\begin{quote}
\begin{multline}
y''(L) = \frac{f(L-L)^2}{2EI} - \frac{gpL}{EI\pi}* [\frac{L}{\pi} (-sin(\frac{\pi}{L}L)) - L + L]
\end{multline}
\end{quote}

\begin{quote}
\begin{equation*}
y''(L) = 0 - \frac{gpL}{EI\pi}* [\frac{L}{\pi} (-sin(\pi)) ]
\end{equation*}
\end{quote}

\begin{quote}
\begin{equation*}
y''(L) = 0 - \frac{gpL}{EI\pi}* [\frac{L}{\pi} (0) ]
\end{equation*}
\end{quote}

\begin{quote}
\begin{equation*}
y''(L) = 0 - 0 = 0
\end{equation*}
\end{quote}
Tredje kriteriet stemmer da y''(L) = 0.
4. Skal nå bevise at fjerde kriteriet, y'''(L) = 0, og starter med å finne den tredje deriverte
\begin{quote}
\begin{multline*}
y''(x) = \frac{f(x-L)^2}{2EI} - \frac{gpL}{EI\pi}(\frac{L^2}{\pi^2} cos(\frac{\pi}{L}x) - \frac{x^2}{2} + Lx - \frac{L^2}{\pi^2})
\end{multline*}
\end{quote}

\begin{quote}
\begin{multline*}
y'''(x) = \frac{f(x-L)^2}{2EI} - \frac{gpL}{EI\pi} \frac{d}{dx}(\frac{L^2}{\pi^2} cos(\frac{\pi}{L}x) - \frac{x^2}{2} + Lx - \frac{L^2}{\pi^2})
\end{multline*}
\end{quote}

\begin{quote}
\begin{equation}
y'''(x) = \frac{f(x-L)}{EI} - \frac{gpL}{EI\pi} (-cos(\frac{\pi}{L}x) - 1)
\end{equation}
\end{quote}
Sjekker så tredje kriteriet, y'''(L) = 0
\begin{quote}
\begin{equation*}
y'''(L) = \frac{f(L-L)}{EI} - \frac{gpL}{EI\pi} (-cos(\frac{\pi}{L}L) - 1)
\end{equation*}
\end{quote}

\begin{quote}
\begin{equation*}
y'''(L) = 0 - \frac{gpL}{EI\pi} (1 - 1)
\end{equation*}
\end{quote}

\begin{quote}
\begin{equation*}
y'''(L) = 0 - \frac{gpL}{EI\pi} 0 = 0
\end{equation*}
\end{quote}
Fjerde kriteriet stemmer fordi y'''(L) = 0.
