\section{Teori}

\subsection{Euler-Bernoulli}
Flere av oppgavene i denne rapporten omhandler Euler-Bernoulli-bjelken. I følge Wikipedia er dette en grunnleggende metode for å regne ut hvor mye en bjelke av et gitt materiale bøyer seg under press. Metoden er oppkalt etter Jacob Bernoulli som gjorde de største oppdagelsene, det var derimot ikke før rundt 1750 at Leonhard Euler og Daniel Bernoulli kom opp med en komplett teori. Metoden ble ikke brukt i praksis på noen større prosjekter før byggingen av Eiffeltårnet i 1889, men har siden blitt en hjørnestein innen ingeniørkunsten.\footnote{\url{https://en.wikipedia.org/wiki/Euler-Bernoulli_beam_theory}, (14.04.2016)} Euler-Bernoulli-likningen er som følger:
\begin{quote}
\begin{equation}
EIy''''=f(x)
\end{equation}
\end{quote}
Den vertikale forskyvningen av en $L$ meter lang bjelke, , oppfyller likningen når $0\leq x\leq L$. Likningen inneholder noen konstanter; $E$ er en materialkonstant kalt Youngmodulusen. $I$ er et arealmoment til bjelkens tverrsnitt normalt på lengderetningen. $I$ er konstant langs hele bjelken. Høyresiden av likningen, $f(x)$, er en kraft som virker på bjelken per lengde-enhet målt i Newton per meter. Kraften inkluderer vekten til bjelken. Dette har vi fra læreboken.\footnote{Sauer, Timothy, (2012) Numerical Analysis second edition, side 102-.}

Hvorfor kan dette brukes på stupebrett?
Legge til ekstra kraft for haug og person.

\subsection{Numerisk kontra Korrekt}

\subsection{Taylors Teorem}
\begin{align*}
 \sum_{k = 0}^{\infty}\frac{f^{(k)}(a)}{k!}(x-a)^{k}
 \end{align*}
 
 \subsection{Kondisjonstall}
Når vi snakker om kondisjonstallet til en funksjon f med et argument x måler vi omfanget av endringer i resultatet til f med liten endring i x. Vi bruker dette til å estimere hvor sensitiv en funksjon er til endringer eller feil i x, og størrelsen på feil som kan forekomme. I matrisesammenheng kan vi se på den lineære likningen Ax = b. Kondisjonstallet til A vil fortelle oss om hvor store endringene i b blir med små endringer i x. I MATLAB benytter vi oss av "cond(A)" for å finne kondisjonstallet. Denne metoden bygger på funksjonen:
 \
 \begin{equation}
 Cond(A) = ||A^{-1}|| \cdot ||A||
 \end{equation}
 
Dersom kondisjonstallet er 1 vil dette bety at en feil i x ikke vil kunne lage større skade enn feilen i seg selv. Så kort sagt er et lavt kondisjonstall bra, og et stort dårlig. 

USIKKER PÅ OM DETTE STEMMER, MÅ SJEKKES I MATLAB!!!:

Gitt matrisen A = $\begin{matrix}
 
	1 & 1/2\\
	1/2 & 1/3
	
\end{matrix}$
\\
og vektorene $b_1 = (2/3, 1)'$ og $b_2 = (2/3, 1)'$
\\Her skulle vi tro at $Ab_1$ og $Ab_2$ ville blitt ganske lik, men med å regne ut finner vi at:\\$x_1 = (0, 3)'$ og $x_2 = (1, 1)'$
\\Med dette ser vi at noen matriser er veldig sensitive til små endringer i inn-data.

\subsection{Feil}
Forover og bakoverfeil.
 
\subsection{MATLAB}
