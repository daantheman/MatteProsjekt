\section{Teori}

\subsection{Eurler-Bernoulli}
Flere av oppgavene i denne rapporten omhandler Euler-Bernoulli-bjelken. I følge Wikipedia er dette en grunnleggende metode for å regne ut hvor mye en bjelke av et gitt materiale bøyer seg under press. Metoden er oppkalt etter Jacob Bernoulli som gjorde de største oppdagelsene, det var derimot ikke før rundt 1750 at Leonhard Euler og Daniel Bernoulli kom opp med en komplett teori. Metoden ble ikke brukt i praksis på noen større prosjekter før byggingen av Eiffeltårnet i 1889, men har siden blitt en hjørnestein innen ingeniørkunsten.\footnote{\url{https://en.wikipedia.org/wiki/Euler-Bernoulli_beam_theory}, (14.04.2016)} Euler-Bernoulli-likningen er som følger:
\begin{quote}
\begin{equation}
EIy''''=f(x)
\end{equation}
\end{quote}
Den vertikale forskyvningen av en $L$ meter lang bjelke, , oppfyller likningen når $0\leq x\leq L$. Likningen inneholder noen konstanter; $E$ er en materialkonstant kalt Youngmodulusen. $I$ er et arealmoment til bjelkens tverrsnitt normalt på lengderetningen. $I$ er konstant langs hele bjelken. Høyresiden av likningen, $f(x)$, er en kraft som virker på bjelken per lengde-enhet målt i Newton per meter. Kraften inkluderer vekten til bjelken. Dette har vi fra læreboken.\footnote{Sauer, Timothy, (2012) Numerical Analysis second edition, side 102-.}
\subsection{Taylors Teorem} 
 