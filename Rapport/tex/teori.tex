\section{Teori}

\subsection{Euler-Bernoulli}
Flere av oppgavene i denne rapporten omhandler Euler-Bernoulli-teoremet. Vi skal blandt annet bruke det til å regne ut bøyningen i et stupebrett, men vi skal også regne litt på den korrekte løsningen av likningen. Først litt generelt om temoremet; I følge Wikipedia er dette en grunnleggende metode for å regne ut hvor mye en bjelke av et gitt materiale bøyer seg under press. Metoden er oppkalt etter Jacob Bernoulli som gjorde de største oppdagelsene, det var derimot ikke før rundt 1750 at Leonhard Euler og Daniel Bernoulli kom opp med en komplett teori. Metoden ble ikke brukt i praksis på noen større prosjekter før byggingen av Eiffeltårnet i 1889, men har siden blitt en hjørnestein innen ingeniørkunsten.\footnote{\url{https://en.wikipedia.org/wiki/Euler-Bernoulli_beam_theory}, (14.04.2016)} Euler-Bernoulli-likningen som vi skal bruke i disse oppgavene er som følger:
\begin{quote}
\begin{equation}
EIy''''=f(x)
\end{equation}
\end{quote}
Den vertikale forskyvningen av en $L$ meter lang bjelke, oppfyller likningen når $0\leq x\leq L$. Likningen inneholder noen konstanter; $E$ er en materialkonstant kalt Youngmodulusen. $I$ er et arealmoment til bjelkens tverrsnitt normalt på lengderetningen. $I$ er konstant langs hele bjelken. Høyresiden av likningen, $f(x)$, er en kraft som virker på bjelken per lengde-enhet målt i Newton per meter. Kraften inkluderer vekten til bjelken. Dette har vi fra læreboken. Vi skal i denne oppgaven kun bruke konstanter som er oppgitt i læreboka.\footnote{Sauer, Timothy, (2012) Numerical Analysis second edition, side 102-.}

Dette teoremet passer bra til et stupebrett, da vi kan se på dette som en bjelke som blir bøyd på grunn av last. Lastene vi skal se på senere i denne oppgaven er først en sinusformet haug og deretter en person. Da må vi utvide formelen vår ved å legge til følgende funksjoner til kraft-delen av likningen:
\begin{quote}
\begin{equation}
s(x)=-pg\;sin\frac{\pi}{L}x
\end{equation}
\begin{equation}
s_2(x)=\begin{cases}
	-g\cdot{50/0.3kg/m} & ,L-0.3m\leq x\leq L\\
	0N/m & ,0m\leq x\leq L-0.3m
	\end{cases}
\end{equation}
\end{quote}
Vi legger til $s(x)$ for den sinusformede haugen og $s_2(x)$ for personen. 

\subsection{Taylors Teorem}

\subsection{Feil}
Når vi snakker om feil i nummerikk, snakker vi om nøyaktigheten i utregninger og hvor nært et riktig svar vi kan komme. Det er ofte snakk om hvor mange desimaler vi kan få riktig. Når vi regner med maskin vil vi grunnet avrundingen i flyttall få en maskin-feil epsilon som er $\epsilon_{mach} = 2^{-52}$. Dette gjør at det blir nesten umulig å regne med mer enn 16 desimalers nøyaktighet. Vi deler feil opp i bakover- og foroverfeil. Enkelt forklart er bakoverfeil feil med "inputen" til et problem, mens foroverfeil er feil med "outputen".

\subsubsection{Foroverfeil}
I vår oppgave jobber vi mest med matriselikninger og da har vi følgende definisjon på foroverfeil; Hvis vi for eksempel har matriselikningen $Ax = b$ med tilnærmet løsning $x_a$, får vi følgende definisjoner: 
\begin{quote}	
\begin{equation*}
\mbox{Foroverfeil} = ||x - x_a||_{\infty}
\end{equation*}
\begin{equation*}
\mbox{Relativ foroverfeil} = \frac{||x - x_a||_{\infty}}{||x||_{\infty}}
\end{equation*}
\end{quote}
Her vil $x$ og $x_a$ være vektorer/matriser, og det vi gjør er å ta uendelighetsnormen av dem. Å ta uendelighetsnormen betyr at vi finner den største absoluttverdien av alle radsummene i matrisen. I tilfeller hvor disse er vektorer blir det da største absoluttverdi i vektoren. På én oppgave finner vi også 1-normen av en matrise. Da finner vi summen av absoluttverdien av alle radsummene.

\subsubsection{Bakoverfeil}
Siden vi jobber mest med matriselikninger får vi følgende definisjon på bakoverfeil; Hvis vi for eksempel har matriselikningen $Ax = b$ med tilnærmet løsning $x_a$ og Residualet: $r = b - Ax_a$, får vi følgende definisjoner:
\begin{quote}	
\begin{equation*}
\mbox{Bakoverfeil} = ||r||_{\infty} = ||b-Ax_a||_{\infty}
\end{equation*}
\begin{equation*}
\mbox{Relativ bakoverfeil} = \frac{||r||_{\infty}}{||b||_{\infty}}
\end{equation*}
\end{quote}

\subsubsection{Feilforstørring}
Når vi regner med feil på denne måten kan det være greit å se hvordan bakoverfeil og foroverfeil påvirker hverandre. Feilforstørringsfaktoren er et slikt tall som viser sammenhengen mellom relativ foroverfeil og relativ bakoverfeil. Dette tallet regnes ut slik:
\begin{quote}
\begin{equation*}
\mbox{Feilforstørring} = \frac{\mbox{Relativ foroverfeil}}{\mbox{Relativ bakoverfeil}} = \frac{\frac{||x - x_a||_{\infty}}{||x||_{\infty}}}{\frac{||r||_{\infty}}{||b||_{\infty}}}
\end{equation*}
\end{quote}
 
\subsection{MATLAB}
For å løse de vanskeligste problemene i dette prosjektet, trenger vi hjelp fra et digitalt verktøy. Og vi bruker MatLab. MatLab, kort for Matrix Laboratory, er et omfattende matematikk-program utiklet av MathWorks. Siden programmet først ble utgitt i 1984 har det blitt et veldig populært program til bruk på utdanningsinstitusjoner. I 2004 hadde MatLab over 1 million registrerte lisenser på verdensbasis.
MatLab opererer med et eget script-språk basert på programmeringsspråket C, så det er veldig nyttig for oss dataingeniører. Med MatLab kan vi utføre matriseregning, behandle og plotte funksjoner, implementere algoritmer osv. Det at det opererer på et script-språk gjør også at vi kan skrive egne funksjoner for å løse problemer. Ved å skrive kode for å løse oppgavene er det lett for oss å vise fremgangsmåte i tillegg til riktig svar.

\subsubsection{Noen MATLAB-metoder vi bruker}
\textbf{hentKonstanter()}\\
Dette er en funksjon vi har laget selv. Denne bruker vi i flere av oppgavene våre. Det den gjør er å hente ut de konstantene vi har fått definert fra læreboka slik at vi slipper å definere nye variabler for hver gang de skal brukes. På denne måten har vi alltid variablene på plass når vi trenger dem.\\
\\
\textbf{repmat}\\
Brukes til å repetere n antall kopier av matrisen.\\
\\
\textbf{ones}\\
Lager en matrise av enere.\\
\\
\textbf{sparsediag}\\
Henter ut diagonale vektorer for ei sparsematrise. Hvor det ikke bare er 0’ere.\\
\\
\textbf{Zeros}\\
Lager en matrise av bare 0’ere.\\
\\
\textbf{abs(n)}\\
Tar absoluttverdien av n.\\
\\
\textbf{Table}\\
Brukes for å lage en tabell av input.\\
\\
\textbf{Plot}\\
Lager et vindu med for eksempel plot(X,Y) der dataen i Y korresponderer mot verdien i X.\\
\\
\textbf{Subplot}\\
Lager en matrise av plotfunksjonen med bestemt rad og rekke.