% latexmal.tex - Mal beregnet for bruk i INF1080
% Time-stamp: <2012-08-27 11:08:39>
%%%%%%%%%%%%%%%%%%%%%%%%%%%%%%%%%%%%%%%%%%%%%%%%%%%%%%%%%%%%%%%%%%%%%%%%%%%
%
% Dette dokumentet har innstillinger som fungerer på Ifi-serverne.
% Du kan lage en pdf-fil av denne filen med:
%  pdflatex latexmal.tex
%
% Et LaTeX-dokument har to deler. Først skriver du inn ting som
% gjelder hele dokumentet, blant annet hvilke pakker du skal bruke.
% Så kommer selve innholdet, mellom \begin{document} og \end{document}
%
% Husk at tegnene: # $ % ^ & _ { } ~ og \ har spesiell betydning.
%
% For mer hjelp og mer info om hvordan du kan stille inn dokumentet
% er disse sidene bra:
% http://www.mn.uio.no/ifi/tjenester/it/hjelp/latex/
% http://en.wikibooks.org/wiki/LaTeX/
%
%%%%%%%%%%%%%%%%%%%%%%%%%%%%%%%%%%%%%%%%%%%%%%%%%%%%%%%%%%%%%%%%%%%%%%

% 1. Hva slags dokument.
\documentclass[12pt,norsk,a4paper]{article}
\usepackage{float}

\usepackage{hyperref}
\usepackage[all]{hypcap}

\hypersetup{
    colorlinks,
    citecolor=black,
    filecolor=black,
    linkcolor=black,
    urlcolor=blue
}

%\usepackage{apacite}
%\usepackage{mathtools}
\usepackage{amsmath}
% For eget tittelblad: legg ,titlepage til rett etter a4paper
% For tosidig: legg til ,twopage


% 3. Norske tegn og norsk utseende

% 3a. Tekstfiler med norske tegn kan lagres i utf-8 eller iso-latin-1
% (iso-8859-1). På Ifi-maskinene er iso-8859-1/15 standard.
\usepackage[utf8]{inputenc} % evt utf8 i stedet for latin1
%\DeclareUnicodeCharacter{030A}{~}

% Please add the following required packages to your document preamble:
\usepackage[table,xcdraw]{xcolor}
% If you use beamer only pass "xcolor=table" option, i.e. \documentclass[xcolor=table]{beamer}
\usepackage{listings}
\usepackage[title]{appendix}
% 2. Om dokumentet - brukes blant annet til tittelen
\title{Numerical analysis - Matteprosjekt}
\author{Sondre Løvhaug, Dan Rasmussen,\\
	Eiric Halland Wærness, Håvard Granaune Matberg
}
\date{\today}


% 3b. For norsk orddeling og dato
\usepackage[norsk]{babel}
\usepackage[norsk]{isodate}

% 3c. For norske tegn
\usepackage[T1]{fontenc}

% 3d. For parallelle avsnitt (fint i INF1080, men ikke så fint i vanlige artikler)
\usepackage{parskip}

% 3e. For å formattere URL-er.
%\usepackage{url}

% 4. Skrifttype og symboler
\usepackage[small,euler-digits]{eulervm}
%\usepackage[bitstream-charter]{mathdesign}
\usepackage{color}

% 5. For litt mindre marger enn standard i LaTeX
\usepackage{a4wide}
\usepackage[top=2cm, margin=2.5cm]{geometry}

% 6. Figurer og bilder

% 6a. For å bryte tekst rundt en figur
%\usepackage{wrapfig}

% 6b. For bilder - med denne pakken kan du legge inn bilder og
% illustrasjoner slik:
%                  \includegraphics[width=.5\textwidth]{bilde.pdf}
\usepackage{graphicx}
\graphicspath{ {img/} }

% 7. For grafer og slikt
%\usepackage{tikz}
%\usetikzlibrary{trees}
% Det fins mange tikz-bibliotek, det er nesten ingen grenser for hva
% du kan lage med dette. Se http://www.texample.net/tikz/

% 8. Egendefinerte kommandoer kan du legge inn slik:
\newcommand{\tuple}[1]{\ensuremath{\langle #1\rangle}}
\newcommand{\set}[1]{\ensuremath{\{#1\}}}
\newcommand{\imp}{\ensuremath{\rightarrow}}
\newcommand{\M}{\ensuremath{\mathcal{M}}}
\newcommand{\AF}{edge from parent[draw=none]}
\newcommand{\NODE}[1]{\{node \{\ensuremath{#1}\}\}}
% http://en.wikibooks.org/wiki/LaTeX/Customizing_LaTeX#New_commands
\usepackage{textcomp}

% 9. Pakke for å formatere Matlab-dokumneter:

\usepackage[]{mcode}


%%%%%%%%%%%%%%%%%%%%%%%%%%%%%%%%%%%%%%%%%%%%%%%%%%%%%%%%%%%%%%%%%%%%%%%%%%%
% Selve innholdet:
\begin{document}

% a. Lager en tittel på dokumentet.
\maketitle
% a. For å få bokstavnummerering på 'subsection's
%\renewcommand{\thesubsection}{(\alph{subsection})}
%\vspace{20mm}

\newpage
\textbf{Revisjonshistorie} \\
\begin{tabular}{|l|l|l|l|}
    \hline
    \textbf{Dato} & \textbf{Versjon} & \textbf{Beskrivelse} & \textbf{Forfatter} \\
    \hline
    26 Januar 2016 & 0.1 & Første versjon & ESN \\
    \hline
\end{tabular}
\newpage
\tableofcontents
%\listoffigures
%\listoftables
\newpage

\section{Innledning}
Gruppe 11 ble sammensatt av at vår faglærer teamet sammen Sondre Løvhaug og Dan Rasmussen sammen med Håvard Matberg og Eric Wærness. Vi har ikke jobbet sammen på tidligere prosjekt, men det virker som gruppa ble en fin match. Etter første møte med faglærer gikk vi løst på oppgavesettet. Ved første øyekast kunne det virke noe utfordrende, men med den ustoppelige kraften fra samarbeid fikk vi raskt kommet i gang med, og løst, første oppgave.
\newpage
\section{Resultater}
\newpage
\section{Referanseliste}

\newpage
\section{Vedlegg}

\subsection{Oppgave 1}
\subsubsection{Oppgave 5.1.21}
Prove the second-order formula for the fourth derivative
\begin{quote}
\begin{equation}\label{eq:oppgave1}
f^{(4)} (x) = \frac{f(x-2h) - 4f(x-h) + 6f(x) - 4f(x+h) + f(x+2h)}{h^4} + O(h^2)
\end{equation}
\end{quote}
Ifølge Taylor's teorem, dersom f er fem ganger kontinuerlig deriverbar, 
kan vi bruke f(x + h) og f(x -h)
\begin{quote}
\begin{equation} \label{eq:f(x+h)}
f(x+h) = f(x) + hf'(x) + \frac{h^2}{2} f''(x) + \frac{h^3}{6} f'''(x) + \frac{h^4}{24} f^4 (x) + \frac{h^5}{120} f^5 (x) + O(h^6)
\end{equation}
\end{quote}
\begin{quote}
\begin{equation} \label{eq:f(x-h)}
f(x-h) = f(x) - hf'(x) + \frac{h^2}{2} f''(x) - \frac{h^3}{6} f'''(x) + \frac{h^4}{24} f^4 (x) - \frac{h^5}{120} f^5 (x) + O(h^6)
\end{equation}
\end{quote}
Legger så sammen f(x+h) + f(x-h) for å eliminere odde-talls deriverte:

\begin{multline} \label{eq:f(x+h)+f(x-h)}
f(x+h) + f(x-h) = \\f(x) + hf'(x) + \frac{h^2}{2} f''(x) + \frac{h^3}{6} f'''(x) + \frac{h^4}{24} f^4 (x) + \frac{h^5}{120} f^5 (x) + O(h^6) +\\
 f(x) - hf'(x) + x\frac{h^2}{2} f''(x) - \frac{h^3}{6} f'''(x) + \frac{h^4}{24} f^4 (x) - \frac{h^5}{120} f^5 (x) + O(h^6) \\ = 2f(x) + h^2 f''(x) + \frac{h^4}{12} f^{(4)} (x) + O (h^6)
\end{multline}
I følge Taylor's teorem, dersom f er fem ganger kontinuerlig deriverbar, kan vi bruke f(x+2h) og f(x-2h). \\
Legger så sammen f(x+2h) + f(x-2h) for å eliminere oddetalls-deriverte. \\
Siden vi allerede har regnet ut f(x+h) + f(x-h), setter vi inn for f(x+2h) og f(x-2h) og får:
\begin{equation} \label{eq:f(x+2h)+f(x-2h)}
f(x+2h) + f(x-2h) = 2f(x) + 2h^2 f(x) + \frac{4h^4}{3} f^4 (x) + O (h^6)
\end{equation}
For å få den fjerde-deriverte aleine, må vi eliminere den andre-deriverte. Dette gjøres ved å gange inn 4 i den første likningen (\ref{eq:f(x+h)+f(x-h)}) og trekker fra likning (\ref{eq:f(x+2h)+f(x-2h)}): 

\begin{multline*}
4 * (f(x+h) + f(x-h)) = 8f(x) + 4h^2 f''(x) + \frac{4h^4}{12} f^4 + 4O(h^6) \\
\Downarrow
\\
4f(x+h) + 4f(x-h) - (f(x+2h) + f(x-2h)) = \\ 
8f(x) + 4h^2 f''(x) + \frac{4h^4}{12} f^4 (x) + 4O(h^6) - \\
2f(x) + 2h^2 f(x) + \frac{4h^4}{3} f^4 (x) + O (h^6) \\
= 6f(x) - h^4 f^4 (x) + 3O (h^6)
\end{multline*}
Snur litt om på likningen og får at:
\begin{multline}
h^4 f^4 = -4f(x+h) - 4f(x-h) + f(x-2h) + f(x-2h) + 6f(x) + 3O(h^6)\\
\Rightarrow f^4 (x) = \frac{f(x-2h) - 4f(x-h) + 6f(x) - 4f(x+h) + f(x+2h)}{h^4} + O(h^2)
\end{multline}


\subsubsection{Oppgave 5.1.22a}
Prove that if f(x) = f'(x) = 0, then
\begin{quote}
\begin{equation}\label{eq:oppgave2}
f^{(4)} (x + h) - \frac{16f(x + h) - 9f(x + 2h) + \frac{8}{3}f(x + 3h) - \frac{1}{4}f(x + 4h)}{h^4} = O(h^2)
\end{equation}
\end{quote}
Starter med å bevise at dersom f(x) = f'(x) = 0, så er
\begin{quote}
\begin{equation}
f(x + h) = 10f(x+h) + 5f(x + 2h) - \frac{5}{3}f(x + 3h) - \frac{1}{4}(x+4h) = O(h^6)
\end{equation}
\end{quote}
Fra oppgave 1, 5.1.21 har vi (\ref{eq:oppgave1}), vi begynner med å skrive denne om til f(x+h) og får:
\begin{quote}
\begin{equation}\label{eq:f^4(x+h)}
f^{(4)}(x+h) = \frac{-4f(x) + f(x-h) + 6f(x+h) - 4f(x+2h) f(x+3h}{h^4} + O(h^2)
\end{equation}
\end{quote}

Nå har vi en likning for $ f^{(4)} (x+h)$ og kan sette denne inn i (\ref{eq:oppgave2})


\begin{multline*}
\frac{-4f(x) + f(x-h) + 6f(x+h) - 4f(x+2h) + f(x+3h)}{h^4} + O(h^2)\\
-
\\
\frac{16f(x+h) - 4f(x+2h) + \frac{8}{3}f(x+3h) - \frac{1}{4}f(x+4h)}{h^4}  = O(h^2)
\\
\Downarrow
\\
\frac{-4f(x)+f(x-h)-10f(x+h)+5f(x+2h)-\frac{5}{3}f(x+3h)+\frac{1}{4}f(x+4h}{h^4} + O(h^2) = O(h^2)
\end{multline*}
Fra oppgaveteksten har vi oppgitt at:
\begin{equation*}
f(x-h)-10f(x+h)+5f(x+2h)-\frac{5}{3}f(x+3h)+\frac{1}{4}f(x+4h) = O(h^6)
\end{equation*}

Vi setter dette inn i likninger vår og får:

\begin{equation*}
\frac{-4f(x) + O(h^6)}{h^4} + O(h^2) = O(h^2)
\end{equation*}

Siden f(x) = f'(x) = 0 ender vi med:


\begin{equation*}
\frac{-0+ O(h^6)}{h^4} + O(h^2) = O(h^2)
\end{equation*}
\begin{equation*}
\Downarrow
\end{equation*}
\begin{equation*}
O(h^2) + O(h^2) = O(h^2)
\end{equation*}

\subsection{Oppgave 6}
\subsubsection{Oppgave 6a}
Vi legger til en funksjon i Euler-likningen
\begin{quote}
$s(x) = -pgsin\frac{\pi}{L} x $
\end{quote} 
til kraftdelen til f(x)
\begin{quote}
$EIy^{(4)} = f(x) + s(x)$
\end{quote}
Skal bevise at
\begin{quote}
\begin{equation}
y(x) = \frac{f}{24EI} x^2 (x^2 - 4Lx + 6L^2) - \frac{gpL}{EI\pi} (\frac{L^3}{\pi^3} sin\frac{\pi}{L}x - \frac{x^3}{6} + \frac{Lx^2}{2} - \frac{L^2 x}{\pi^2})
\end{equation}
\end{quote}
tilfredstiller Euler-Bernoullie likningen og randbetingelsene for en bjelke som er fast i den ene enden og fri i den andre\\
$y(0) = y'(0) = y''(L) = y'''(L) = 0$ \\
\\
1. Starter med å bevise at y(0)= 0
\begin{quote}
\begin{multline*}
y(0) = \frac{f}{24EI} 0^2 (0^2 - 4L0 + 6L^2) - \frac{gpL}{EI\pi} (\frac{L^3}{\pi^3} sin\frac{\pi}{L}x - \frac{0^3}{6} + \frac{L0^2}{2} - \frac{L^2 0}{\pi^2}) \\
y(0) = 0 - \frac{gpL}{EI\pi} (\frac{L^3}{\pi^3} sin (0) - 0 + 0 - 0 ) \\
y(0) = 0
\end{multline*}
\end{quote}
Første kriterie er oppfylt.
\\
2. Skal nå bevise kriterie 2, at y'(0) = 0, starter med å finne den deriverte.
\begin{quote}
\begin{multline}
y'(x) = \frac{d}{dx} [\frac{fx^2(x^2-4Lx+6L^2)}{24EI} - \frac{gpL}{EI\pi} (\frac{L^3}{\pi^3} sin \frac{\pi}{L}x - \frac{x^3}{6} + \frac{Lx^2}{2} - \frac{L^2x}{\pi^2})]
\end{multline}
\end{quote}

\begin{quote}
\begin{multline*}
y'(x) = \frac{fx(x^2-3Lx+3L^2)}{6EI} - \frac{gpL}{EI\pi}*\frac{d}{dx} (\frac{L^3}{\pi^3} sin \frac{\pi}{L}x - \frac{x^3}{6} + \frac{Lx^2}{2} - \frac{L^2x}{\pi^2})
\end{multline*}
\end{quote}

\begin{quote}
\begin{multline*}
y'(x) = \frac{fx(x^2-3Lx+3L^2)}{6EI} - \frac{gpL}{EI\pi} (\frac{L^2}{\pi^2} cos(\frac{\pi}{L}x) - \frac{x^2}{2} + Lx - \frac{L^2}{\pi^2})
\end{multline*}
\end{quote}
Sjekker om y'(0) = 0
\begin{quote}
\begin{multline*}
y'(0) = \frac{f0(0^2-3L0+3L^2)}{6EI} - \frac{gpL}{EI\pi} (\frac{L^2}{\pi^2} cos(\frac{\pi}{L}0) - \frac{0^2}{2} + L0 - \frac{L^2}{\pi^2})
\end{multline*}
\end{quote}

\begin{quote}
\begin{equation*}
y'(0) = 0 - \frac{gpL}{EI\pi} (\frac{L^2}{\pi^2} * 1 -\frac{L^2}{\pi^2}) \\
\end{equation*}
\end{quote}

\begin{quote}
\begin{equation*}
y'(0) = 0 - \frac{gpL}{EI\pi} (0) = 0
\end{equation*}
\end{quote}
Andre kriteriet er oppfylt.
\\
3. Skal nå bevise kriteriet 3, y''(L) = 0. Starter med å finne den andre deriverte.
\begin{quote}
\begin{multline*}
y'(x) = \frac{fx(x^2-3Lx+3L^2)}{6EI} - \frac{gpL}{EI\pi} (\frac{L^2}{\pi^2} cos(\frac{\pi}{L}x) - \frac{x^2}{2} + Lx - \frac{L^2}{\pi^2})
\end{multline*}
\end{quote}
Fra oppgave 4a, vet vi at
\begin{quote}
\begin{multline*}
y''(x) = \frac{f(x-L)^2}{2EI} - \frac{gpL}{EI\pi} \frac{d}{dx}(\frac{L^2}{\pi^2} cos(\frac{\pi}{L}x) - \frac{x^2}{2} + Lx - \frac{L^2}{\pi^2})
\end{multline*}
\end{quote}

\begin{quote}
\begin{multline}
y''(x) = \frac{f(x-L)^2}{2EI} - \frac{gpL}{EI\pi}* [\frac{L}{\pi} (-sin(\frac{\pi}{L}x)) - x + L]
\end{multline}
\end{quote}
Sjekker nå om y''(L)=0
\begin{quote}
\begin{multline}
y''(L) = \frac{f(L-L)^2}{2EI} - \frac{gpL}{EI\pi}* [\frac{L}{\pi} (-sin(\frac{\pi}{L}L)) - L + L]
\end{multline}
\end{quote}

\begin{quote}
\begin{equation*}
y''(L) = 0 - \frac{gpL}{EI\pi}* [\frac{L}{\pi} (-sin(\pi)) ]
\end{equation*}
\end{quote}

\begin{quote}
\begin{equation*}
y''(L) = 0 - \frac{gpL}{EI\pi}* [\frac{L}{\pi} (0) ]
\end{equation*}
\end{quote}

\begin{quote}
\begin{equation*}
y''(L) = 0 - 0 = 0
\end{equation*}
\end{quote}
Tredje kriteriet stemmer da y''(L) = 0.
4. Skal nå bevise at fjerde kriteriet, y'''(L) = 0, og starter med å finne den tredje deriverte
\begin{quote}
\begin{multline*}
y''(x) = \frac{f(x-L)^2}{2EI} - \frac{gpL}{EI\pi}(\frac{L^2}{\pi^2} cos(\frac{\pi}{L}x) - \frac{x^2}{2} + Lx - \frac{L^2}{\pi^2})
\end{multline*}
\end{quote}

\begin{quote}
\begin{multline*}
y'''(x) = \frac{f(x-L)^2}{2EI} - \frac{gpL}{EI\pi} \frac{d}{dx}(\frac{L^2}{\pi^2} cos(\frac{\pi}{L}x) - \frac{x^2}{2} + Lx - \frac{L^2}{\pi^2})
\end{multline*}
\end{quote}

\begin{quote}
\begin{equation}
y'''(x) = \frac{f(x-L)}{EI} - \frac{gpL}{EI\pi} (-cos(\frac{\pi}{L}x) - 1)
\end{equation}
\end{quote}
Sjekker så tredje kriteriet, y'''(L) = 0
\begin{quote}
\begin{equation*}
y'''(L) = \frac{f(L-L)}{EI} - \frac{gpL}{EI\pi} (-cos(\frac{\pi}{L}L) - 1)
\end{equation*}
\end{quote}

\begin{quote}
\begin{equation*}
y'''(L) = 0 - \frac{gpL}{EI\pi} (1 - 1)
\end{equation*}
\end{quote}

\begin{quote}
\begin{equation*}
y'''(L) = 0 - \frac{gpL}{EI\pi} 0 = 0
\end{equation*}
\end{quote}
Fjerde kriteriet stemmer fordi y'''(L) = 0.

\newpage


\bibliographystyle{abbrv}
\bibliography{bib/refs}

\newpage
\begin{appendices}
\label{appendix:script}
\end{appendices}
\end{document}
